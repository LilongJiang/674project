% author: Man Cao, Lilong Jiang
\documentclass{article}
\usepackage[letterpaper]{geometry} \usepackage[utf8]{inputenc}
\usepackage[T1]{fontenc}
\usepackage{amsmath}
\usepackage{hyperref}
\usepackage{relsize}
\usepackage{graphicx}

\newcommand{\code}[1]{\textsf{\smaller\verb~#1~}}

\begin{document}

\title{CSE5243 Assignment 4}
\author{Man Cao(cao.235), Lilong Jiang(jiang.573)}
\maketitle

\section{Work Separation}
Lilong mainly worked on hierachical clustering. Man mainly worked on
K-means. In fact there were a lot of overlapping during the
work, we exchanged various ideas and wrote the this report together.
\section{Input}
After eliminating documents without topics, 11367 documents are left.\\
The input file has the following format:
\begin{verbatim}
{'NEWID':<value>, 'TOPICS':[value1, value2, ...], 'PLACES':[value1, value2, ...]}
{<term1>:<value1>, <term1>:<value2>, ...}
\end{verbatim}
Note that each document corresponds to two lines: the first line contains the
metadata of the document, the second line is the frequency vector.

\section{Metrics}
cosine similarity and jaccard similarity is used in this assignment.

\section{Algorithms}
\subsection{Hierachical Clustering}
\subsubsection{Single Link}
We use single-link to cluster the documents.
\subsubsection{Implemtation Details}
Since the similarity between cluster1 and cluster2 is the same as the similarity between cluster2 and cluster1. We only need to store the upper triangle of the matrix.\\
Also considering the expensive time complexity of finding the min distance in the matrix, we transform the proximity matrix to a proximity list and sort the list before the clustering. In this way, we only check the distance from the last position. The mapping relationships between the documents to the clusters are record in a list and updated every time two clusters are merged.  
\subsection{K-means Clustering}
\section{Evaluation}
\subsection{Scalability}
The time for clustering 2, 4, 8, 16 and 32 clusters with Hierachical Clustering and K-means is show in Fig. 
\subsection{Quality}
\subsubsection{Entropy}
\subsubsection{Skew}
The skew is measured as variance of the cardinalities of different clusters. 
\end{document}
